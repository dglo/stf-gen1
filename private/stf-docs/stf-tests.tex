\documentclass{article}
\usepackage{graphicx}
\usepackage{makeidx}
\usepackage{amssymb}
\usepackage{url}
\usepackage[T1]{fontenc}

\makeindex

\begin{document}

\title{Tests for IceCube DOMs} 
\author{Azriel Goldschmidt \\ Arthur Jones \\ 
  Lawrence Berkeley National Labs } 

\maketitle 
\thispagestyle{empty}

\eject

\tableofcontents

\eject

\section{Scope and purpose of this document}

The first prototypes of IceCube Digital Optical Modules Main Board
(DOMMB) are expected early in January 2003. The first phase of testing
will be mostly voltmeter and oscilloscope driven, looking for possible
shorts or disconnected traces or parts, etc. Passed that point, more
sophisticated tests need to be performed. These involve exercising the
various parts on the DOMMB such as digitizers, discriminators,
pulsers, etc. For these, software needs to be written. This document
describes the tests that should be performed by the test software. We
are aiming for pass/no-pass type of tests. However, some of the data
produced in the testing procedure will need to be preserved and
archived in a database for future reference.  For each test, the
following are listed:

\begin{itemize}
\item A list of the required functionality to run it
\item A list of relevant previous tests
\item The list of input values and controls: their meaning, default
  values and type
\item The list of output values and their meaning/type
\item The procedure of the test and the criteria for pass/no-pass (if
  relevant)
\end{itemize}


Temporary notes:
\begin{itemize}
\item A few fields are defined for each test, Name, Version, Loop
  count (which will have slightly different meaning in each case
  because it lives inside the test, pass/fail output (still need to
  define the case of N/A) and the value when an error occurs, error
  code, error string.
\item Give a list of previous relevant tests.
\item Notes from Bruce, speed of ATWDs 1.5 $\pm$0.05, baseline 100
  $\pm$ 20 ???, amplitude 160 $\pm$ 4
\end{itemize}

\section{Tests}

\subsection{ADC tests}

In this test a single ADC channel (a board voltage or current) is read
out multiple times and the mean, spread, maximum and minimum are
compared to expectations to define a pass-no pass criteria.

\subsubsection*{Previous relevant tests} None   

\subsubsection*{Required functionality}Read all ADC channels 
(8 channels of CS0 currently used in Jan 03 version board)

\subsubsection*{Inputs}
\begin{tabular}{lp{2in}ll}
  Input Variable & Description & Type & Default and Range \\
  \input{ADC-inputs}
\end{tabular}

\subsubsection*{Outputs}
\begin{tabular}{lp{2.2in}l}
  Output Variable & Description & Type \\
  \input{ADC-outputs}
\end{tabular}

\subsubsection*{Procedure}
\begin{itemize}
\item Perform LOOP\_COUNT readouts of the ADC from channel
  ADC\_CHANNEL of the ADC chip (chip-select) ADC\_CHIP and calculate
  the mean ADC\_MEAN\_COUNTS (keeping the running sum in a Unsigned
  Long and dividing by LOOP\_COUNT at the end).
\item Perform additional LOOP\_COUNT readouts of the same ADC and save
  the maximum and minimum values in ADC\_MAX\_COUNTS and
  ADC\_MIN\_COUNTS. Also calculate the standard deviation using the
  previously calculated ADC\_MEAN\_COUNTS:
  
  \[ adc\_rms\_counts =                                                   
  \sqrt{ \frac{1}{loop\_count - 1 } \times \sum{(adc_{j} -
      adc\_mean\_counts) ^{2}}} \]
  
\item Translate the mean, standard deviation and min \& max from
  counts to milliamp/millivolt units using the translations below, and
  fill the corresponding output variables:
  \begin{itemize}
  \item Channel 0
    \[ adc\_xxx\_mvolt\_or\_mamp = \frac{adc\_xxx\_counts}{2} \]
    
  \item Channel 1 (no translation for pressure here)
    \[ adc\_xxx\_mvolt\_or\_mamp = adc\_xxx\_counts \]
    
  \item Channel 2
    \[ adc\_xxx\_mvolt\_or\_mamp = \frac{adc\_xxx\_count \times 2500} 
    {4095 \times \frac{10 + 24.9}{10}} \]
    
  \item Channel 3
\[ adc\_xxx\_mvolt\_or\_mamp = \frac{adc\_xxx\_counts}{20} \]

\item Channel 4
\[ adc\_xxx\_mvolt\_or\_mamp = \frac{adc\_xxx\_counts}{20} \]

\item Channel 5
\[ adc\_xxx\_mvolt\_or\_mamp = \frac{adc\_xxx\_counts}{20} \]

\item Channel 6
\[ adc\_xxx\_mvolt\_or\_mamp = \frac{adc\_xxx\_counts}{20} \]

\item Channel 7
\[ adc\_xxx\_mvolt\_or\_mamp = \frac{adc\_xxx\_counts}{2} \]
\end{itemize}

\item Define a pass/no-pass criteria and set passed accordingly.

\begin{itemize}
\item All Channels
\[ passed = adc\_max\_counts < 4095 \; and \; adc\_min\_counts > 0 \]

\item Channel 0
\[ passed = 200-20 < adc\_mean\_mvolt\_or\_mamp < 200+20 \; and \; \cdots \] 
\[      adc\_rms\_mvolt\_or\_mamp < 20 \; and \; \cdots \]
\[      adc\_max\_mvolt\_or\_mamp < 200+40 \; and \; \cdots \]
\[      adc\_min\_mvolt\_or\_mamp > 200-40 \]

\item Channel 2
\[ passed = 2000-100 < adc\_mean\_mvolt\_or\_mamp < 2000+100 \; and \; \cdots \] 
\[      adc\_rms\_mvolt\_or\_mamp < 20 \; and \; \cdots \]
\[      adc\_max\_mvolt\_or\_mamp < 2000+120 \; and \; \cdots \]
\[      adc\_min\_mvolt\_or\_mamp > 2000-120 \]
\end{itemize}

\end{itemize}

\subsection{DAC test}
\subsection{Flasher Board Interface test}
\subsection{On-board LED Interface test}
\subsection{HV Interface test}
\subsection{PMT Signal Discriminator test}

\subsection{ATWD tests}

\subsubsection*{Previous relevant tests}

\begin{itemize}
\item DAC setting
\end{itemize}

\subsubsection*{Required functionality}

\begin{itemize}
\item Set all ATWD-related DACs for both ATWD chips.
\item Set SPE discriminator threshold DAC.
\item FPGA: Forced launch of ATWD.
\item FPGA: SPE Discriminator launch of ATWD.
\item FPGA: read out single ATWD waveforms.
\end{itemize}

\subsection{ATWD Pedestal Tests}

In this test a number of waveforms from an ATWD chip/channel are
collected to form an average pedestal waveform. The ATWD capture can
be launched by either forcing it or by setting the SPE discriminator
level within the noise band. The average pedestal waveform is then
analyzed and a pass/no-pass is defined based on the range of pedestal
values. The average pedestal waveform can be requested as an output.

\subsubsection*{Inputs}
\begin{tabular}{lp{2in}ll}
Input Variable & Description & Type & Default and Range \\
\input{atwd_pedestal-inputs}
\end{tabular}

\subsubsection*{Outputs}
\begin{tabular}{lp{2.2in}l}
Output Variable & Description & Type \\
\input{atwd_pedestal-outputs}
\end{tabular}

\subsubsection*{Procedure}
\begin{itemize}
\item All five ATWD DAC settings are programmed.
\item The ATWD trigger mask is written according to
  atwd\_trig\_forced\_or\_spe.
\item If the SPE trigger was requested
  (atwd\_trig\_forced\_or\_spe=1), calculate the SPE DAC that
  corresponds to SPE\_DISCRIMINATOR\_UVOLT (see Note on SPE/MPE DAC at
  the end of this document) and program it.
\item Take one waveform for the ATWD\_CHIP\_A\_OR\_B/ATWD\_CHANNEL
  channel requested.
\item Repeat from step D, LOOP\_COUNT times, keeping a waveform that
  is the sum of all.
\item Divide the resulting sum waveform by LOOP\_COUNT to get an
  average waveform.
\item If FILL\_OUTPUT\_ARRAYS=1, fill the output array with the
  average pedestal waveform.
\item Analyze average pedestal waveform:
\begin{itemize}
\item Obtain maximum and minimum value of the average pedestal
  waveform.
\item Set passed to pass value if ALL of the following are true
\item Minimum value of average pedestal waveform > 0
\item Maximum value of average pedestal waveform < 1023
\item Maximum-Minimum < 20 (check with Martin�s criteria)
\end{itemize}
\item Fill the ATWD\_PEDESTAL\_AMPLITUDE output variable with the
  value of Maximum-Minimum from the previous step.
\end{itemize}

\subsection{ATWD Baseline test}

In this test a number of waveforms from an ATWD chip/channel are
collected. For each waveform an average baseline value is calculated
using all the samples. The ATWD capture can be launched by either
forcing it or by setting the SPE discriminator level within the noise
band.  The average baseline values are used to fill a histogram of
this quantity that can be output on request. The mean, standard
deviation and max \& min of the set of average baseline values are
calculated. These calculated values are used to define a pass/no-pass
criterion.

\subsubsection*{Inputs}
\begin{tabular}{lp{2in}ll}
Input Variable & Description & Type & Default and Range \\
\input{atwd_baseline-inputs}
\end{tabular}

\subsubsection*{Outputs}
\begin{tabular}{lp{2.2in}l}
Output Variable & Description & Type \\
\input{atwd_baseline-outputs}
\end{tabular}

\subsubsection*{Procedure}
\begin{itemize}
\item All five ATWD DAC settings are programmed.
\item The ATWD trigger mask is written according to
  ATWD\_TRIG\_FORCED\_OR\_SPE.
\item If the SPE trigger was requested
  (ATWD\_TRIG\_FORCED\_OR\_SPE=1), calculate the SPE DAC that
  corresponds to SPE\_DISCRIMINATOR\_UVOLT (see Note on SPE/MPE DAC at
  the end of this document) and program it.
\item Take one waveform for the ATWD\_CHIP\_A\_OR\_B/ATWD\_CHANNEL
  channel requested.
\item Calculate the baseline value: sum all samples and divide by the
  number of samples (128).  Integer arithmetic is fine.
\item Enter the value in the histogram of baseline values (it is a
  1024-bins histogram).
\item Keep a sum of all baseline values, and a sum of the squares for
  RMS calculation. Also keep the minimum and maximum value.
\item Repeat from step D, LOOP\_COUNT times.
\item If FILL\_OUTPUT\_ARRAYS=1, fill the output array with the
  histogram of baseline values.
\item Compute the mean and the RMS using the running sums (integer
  arithmetic is ok).
\item Fill output variables ATWD\_BASELINE\_MEAN, ATWD\_BASELINE\_RMS,
  ATWD\_BASELINE\_MIN, ATWD\_BASELINE\_MAX.
\item Set passed to pass value if ALL of the following are true:
\begin{itemize}
\item ??? < ATWD\_BASELINE\_MEAN < ??? NEED CALCULATION OF
  LEVELS(WHICH DEPEND ON CHANNEL NUMBER/GAIN)
\item ATWD\_BASELINE\_RMS < ???
\item ATWD\_BASELINE\_MIN > ???
\item ATWD\_BASELINE\_MAX < ???
\end{itemize}
\end{itemize}

\subsection{ATWD Amplitude Response}
\subsection{ATWD Pedestal Sweep}
\subsection{ATWD Cross-talk between slow control functions and ATWD PMT signal channels}
\subsection{ATWD Previous waveform memory}
\subsection{Fast ADC (FADC) test}

\subsection{Pressure Gauge test (work in progress need to add ADC 5V 
  readout and translation and limits with/out temperature reading)}

In this test the ADC channel corresponding to the pressure gauge is
read out multiple times and the mean, spread, maximum and minimum are
compared to expectations to define a pass-no pass criteria.  Whether
the DOMMB is in a sphere and whether it is cold or not will influence
the pressure reading.

\subsubsection*{Previous relevant tests}
None

\subsubsection*{Required functionality}
? Read all ADC channels (8 channels of CS0 currently used in Jan 03
version board) ? If the USE\_TEMPERATURE option is used requires the
ability to read the on-board temperature gauge.
                                                                        
\subsubsection*{Inputs}
\begin{tabular}{lp{2in}ll}
Input Variable & Description & Type & Default and Range \\
\input{pressure-inputs}
\end{tabular}

\subsubsection*{Outputs}
\begin{tabular}{lp{2.2in}l}
Output Variable & Description & Type \\
\input{pressure-outputs}
\end{tabular}


\subsubsection*{Procedure}
\begin{itemize}
\item Perform $loop\_count$ readouts of the ADC corresponding to the
  pressure gauge and of the ADC corresponding to the 5V power supply (
  calculate the mean $adc\_mean\_counts$ (keeping the running sum in a
  Unsigned Long and dividing by $loop\_count$ at the end).
\item Perform additional $loop\_count$ readouts of the same ADC and
  save the maximum and minimum values in $adc\_max\_counts$ and
  $adc\_min\_counts$.  Also calculate the standard deviation using the
  previously calculated $adc\_mean\_counts$:
\[ adc\_rms\_counts =                                                   
\sqrt{\frac{1}{( loop\_count-1)} \times \sum{(adc_j -
    adc\_mean\_counts)^2}} \]
\item Translate the mean, standard deviation and min \& max from
  counts to kiloPascal units using the translation below and fill the
  corresponding output variables:

\[ adc\_xxx\_kpascal  = \frac{adc\_xxx\_count}{2} \]

\item If $use\_temperature$ was set to 1, get a reading of the
  temperature (Units? Need to wait until I implemented the temperature
  test)
\item Define a pass/no-pass criteria and set passed accordingly. Use
  the following table to determine the limits on the different output
  variables (pass is the logical AND of all the conditions for a given
  channel):
\end{itemize}

\subsection{Temperature Sensor test}
\subsection{Analog Signal Noise test}
\subsection{Pulser tests}
A series of important tests are to be performed with the pulser that
injects PMT-like signals before the delay line and discriminator
circuitry. The pulse amplitude is controlled by a DAC and it is fired
by the FPGA. The pulse shape is determined by the hardware.

\subsubsection*{Required Functionality}
\begin{itemize}
\item DAC: correct voltage output (see DAC test).
\item Pulser: correct voltage output, shape and linearity as a
  function of DAC setting.
\item FPGA: ability to continuously generate pulses at repetition rate
  of ~1kHz. (A selectable repetition rate in the 100Hz-100kHz range
  would be desirable, with a default to ~1kHz if not explicitly set.)
\end{itemize}
   
\subsection{Pulser SPE/MPE discriminator test (Need to be 
  reformatted with style of newest tests)}

\subsubsection*{Required Functionality}
FPGA: SPE/MPE discriminator counters. The counting gate should be
about one second (preferably exactly one second). The discriminator
should be allowed to fire again only after ~5 microseconds. (A
selectable dead-time would be desirable, which defaults to ~5
microseconds if not set.)

\subsubsection*{Inputs}
\begin{itemize}
\item Pulse height in MICROVOLTS (at the PMT input). Unsigned Long.
  Default = 1000 ( = 1mV ~ 0.2 SPE)
\item Discriminator level in MICROVOLTS (�defined� at the PMT
  input). Signed Long. Default = 500 (= 0.5 mV ~ 0.1 SPE)
\item SPE or MPE discriminator. Boolean (0 = SPE, 1=MPE). Default = 0
  (SPE).
\item Pedestal level in MICROVOLTS. Unsigned Long. Default = 3,000,000
  (=3V)
\item Pulser repetition rate in Hz. Unsigned Integer. Default = 0
  (this should choose the FPGA default of ~1kHz).
\item Maximum rate deviation (absolute value) from expected value in
  Hz. Unsigned Integer.  Default = 10.
\end{itemize}

\subsubsection*{Outputs}
\begin{itemize}
\item The measured rate in Hz. Unsigned Long.
\item Test Pass/No-Pass. Boolean.
\end{itemize}

\subsubsection*{Procedure}
The test proceeds as follows:
\begin{itemize}
\item The amplitude of the pulses is used to calculate the value of
  the PULSER DAC.
\item The PULSER DAC is programmed with the target value calculated in
  the previous step.
\item The PEDESTAL DAC is calculated from the input Pedestal value.
\item The PEDESTAL DAC is programmed to the value from the previous
  step.
\item The discriminator level is used to calculate the discriminator
  SPE/MPE DAC.
\item The SPE/MPE DAC is programmed with the target value from the
  previous step.
\item The pulser is set to fire at the nominal ~1kHz rate (or
  different rate if Pulser repetition rate is given).
\item The SPE/MPE counter is read; this step involves starting the
  counter and reading-out the value after the counting gate is
  finished.
\item Check whether the rate from the SPE/MPE counter is consistent
  with expectations; decide pass/no-pass.
\item Turn off pulser (This depends on whether subsequent tests are
  run from clean or not).
  
\item A single parameter is used to determine pass/no pass criteria:
  an absolute deviation from the nominal expected rate. That is, for
  values of the pulser amplitude larger than the threshold the rate
  should be consistent the full rate (in the default case in the
  [990-1010] Hz range) . For values of the pulser smaller than the
  threshold the rate should be consistent with zero rate ([0-10] Hz in
  the default case).
\end{itemize}

\subsubsection*{Notes}
Notes on DAC settings, reference voltage, SPE amplitude and
discriminator settings:
\begin{itemize}
\item A 10-bit DAC channel sets the pulser amplitude level. Its range
  is [0-5V], or 4.88 mV/LSB.  To compute the PULSER DAC use :

\[ PULSER DAC = \frac{1}{5000000} \times \frac{pulse\_in\_microvolts \times 1024 \times (R1+R2)}{R2} \]
[where $R1$ and $R2$ are still not determined but should be around
$\frac{R1+R2}{R2} = 10$ to cover the 0.1-100 SPE range ]
\item A 12-bit DAC channel sets the value of the pedestal. Its range
  is [0-5V]. The value is chosen to provide the ATWD with its ~3V
  target Vref. To compute the PEDESTAL DAC use:

\[ PEDESTAL DAC = \frac{Pedestal\_in\_microvolts  \times 4096}{5,000,000} \] (is this a problem for 
LONGS, or should I think in float calculation?).
\item The nominal magnitude of a single photoelectron pulse out of the
  transformer is 5mV (mean).  Before the discriminators there is a
  gain of ~9.6. Therefore the amplitude is ~48 mV (mean) for the SPE
  at the discriminator.
\item A 10-bit DAC channel sets the SPE/MPE discriminator level. Its
  range is [0-5V], or 1.22 mV/LSB.  To calculate the $spe\_dac$ use:

\[ spe\_dac = ( discriminator\_in\_microvolts \times 9.6 \times
\frac{2200+1000}{1000} + \frac{pedestal\_dac \times 5,000,000}{4096} )
\cdots \]
\[ \times \frac{1024}{5,000,000} \]

To calculate the $mpe\_dac$ use:

\[ mpe\_dac = (discriminator\_in\_microvolts \times 9.6 + 
\frac{pedestal\_dac \times 5,000,000}{4096}) \times
\frac{1024}{5,000,000} \]

\end{itemize}

\subsection{Pulser SPE/MPE discriminator scan test  (Need to be 
  reformatted with style of newest tests)}

In this test the pulser is set to some amplitude and the SPE/MPE
discriminator is scan and the rate at each point is recorded. It is
NOT a pass/no-pass test, but the results should be recorded to a
database.

\subsubsection*{Required Functionality}

FPGA: SPE/MPE discriminator counters. The counting gate should be
about one second (preferably exactly one second). The discriminator
should be allowed to fire again only after ~5 microseconds. (A
selectable �dead-time� would be desirable, which defaults to ~5
microseconds if not set.)

\subsubsection*{Inputs}
\begin{itemize}
\item Pulse height in MICROVOLTS (at the PMT input). Unsigned Long.
  Default = 1000 ( = 1mV ~ 0.2 SPE)
\item Lower value of Discriminator level to scan in MICROVOLTS
  (defined at the PMT input).  Signed Long. Default = -1000 (= -1.0 mV
  ~ -0.2 SPE).
\item Upper value of Discriminator level to scan in MICROVOLTS
  (defined at the PMT input).  Signed Long. Default = 2000 (= 2.0 mV ~
  -0.4 SPE).
\item Step size of scan in MICROVOLTS (�defined� at the PMT
  input).  Unsigned Long. Default = 0 (when this value is zero the
  steps are single DAC setting steps).
\item SPE or MPE discriminator. Boolean (0 = SPE, 1=MPE). Default = 0
  (SPE).
\item Pedestal level in MICROVOLTS. Unsigned Long. Default = 3,000,000
  (=3V)
\item Pulser repetition rate in Hz. Unsigned Integer. Default = 0
  (this should choose the FPGA default of ~1kHz).
\end{itemize}

\subsubsection*{Outputs}
\begin{itemize}
\item An array of measured rates in Hz. Array of Unsigned Long
  (maximum length 1024 if code protects against step size smaller that
  DAC granularity).
\end{itemize}

\subsubsection*{Procedure}
The test proceeds as follows:
\begin{itemize}
\item The amplitude of the pulses is used to calculate the value of
  the PULSER DAC.
\item The PULSER DAC is programmed with the target value calculated in
  the previous step.
\item The PEDESTAL DAC is calculated from the input Pedestal value.
\item The PEDESTAL DAC is programmed to the value from the previous
  step.
\item The pulser is set to fire at the nominal ~1kHz rate (or
  different rate if Pulser repetition rate is given).
\item A loop over the discriminator levels between Lower and Upper
  with the step size requested (for the default values this should
  scan over ~20 interesting points in the electronic noise and true
  discriminator region, which should take about 20 seconds). At each
  point, the following three steps are executed:
\item Calculate the discriminator SPE/MPE DAC.
\item The SPE/MPE DAC is programmed with the target value from the
  previous step.
\item The SPE/MPE counter is read; this step involves starting the
  counter and reading-out the value after the counting gate is
  finished.
\item Turn off pulser (This depends on whether subsequent tests are
  run from clean or not).
\end{itemize}

\subsection{Pulser ATWD0/1/2 test}

In this test the pulser is set to fire at the nominal rate.  A
waveform for a given chip/channel is taken. If the pedestal-subtracted
option is requested, a waveform is taken with the pulser amplitude set
to zero, and it is subtracted from the first waveform.  A number of
these (sometimes pedestal subtracted) waveforms are averaged. The
resulting average waveform (pedestal subtracted if requested) is
analyzed to check if it has the expected pulse width, height and
baseline. The assumption is that the values for the ATWD DAC settings
have been correctly set for this �lot� of ATWDs. Therefore, the
variations in sampling speed and gain (and hopefully also baseline, in
the case of non-pedestal subtracted) are small between chips and
therefore a pass no pass can be defined. This test checks also the
input amplifiers to the various ATWD channels.
  
\subsubsection*{Previous relevant tests}
(These will be specified as one of the tests in this list with
specific input parameters-)
\begin{itemize}
\item ATWD Sampling speed test
\item ATWD Pedestal/baseline test
\item DAC setting
\item Others?
\end{itemize}

\subsubsection*{Required functionality}

(Some of these could be replaced by having passed the relevant
previous tests)

\begin{itemize}
\item Set all ATWD-related DACs for both ATWD chips.
\item FPGA: Launch ATWD synchronously on pulse generation (with
  tunable delay in 2x clocks, defaulting to a delay similar to Delay
  line, if zero it should mean that the pulser fires 4 clock cycles
  before the ATWD).
\item FPGA: read out ATWD waveforms.
\end{itemize}

\subsubsection*{Inputs}
\begin{tabular}{lp{2in}ll}
Input Variable & Description & Type & Default and Range \\
\input{atwd_pulser-inputs}
\end{tabular}

\subsubsection*{Outputs}
\begin{tabular}{lp{2.2in}l}
Output Variable & Description & Type \\
\input{atwd_pulser-outputs}
\end{tabular}

\subsubsection*{Procedure}
The test proceeds as follows:
\begin{itemize}
\item All five ATWD DAC settings are programmed.  M. Set the trigger
  masks to trigger synchronously to the pulse generation (modulo
  delay).
\item Set the pulse delay from the PULSE\_DELAY\_WRT\_LAUNCH input.
\item Set the pulser repetition rate to the default value (writing a
  zero to the corresponding register).
\item If pedestal subtraction is requested set the pulser height to
  zero microvolts.
\item If pedestal subtraction is requested, take one waveform for the
  ATWD\_CHIP\_A\_OR\_B/ ATWD\_CHANNEL channel requested.
\item The PULSER\_AMPLITUDE\_UVOLT input is used to calculate the
  value of the PULSER DAC [need engineer�s input].
\item The PULSER DAC is programmed with the target value calculated in
  the previous step.
\item Take one waveform for the ATWD\_CHIP\_A\_OR\_B/ ATWD\_CHANNEL
  channel requested.
\item If pedestal subtraction is requested, subtract the first wave
  from the second (result is Signed).
\item Repeat from step E, LOOP\_COUNT times, keeping a waveform that
  is the sum of all.
\item Turn off pulser.
\item Divide the resulting sum waveform by LOOP\_COUNT to get an
  average waveform.
\item Perform analysis on waveform to determine baseline, width,
  height and position.
\item To calculate the baseline, make three separate averages of
  samples [60-79], [80-99] and [100-119] and pick the smallest of the
  three as the ATWD\_WAVEFORM\_BASELINE value.
\item Find the position ATWD\_WAVEFORM\_POSITION and value of the
  waveform maximum point.
\item Calculate the ATWD\_WAVEFORM\_HEIGHT subtracting the baseline
  from the maximum amplitude.
\item Calculate the ATWD\_WAVEFORM\_WIDTH (basically a FWHM) by going
  to the left and right of the maximum amplitude position and finding
  the first sample that is smaller or equal half of the pulse height
  (plus the baseline). Subtract the left position from the right
  position.
\item Check if the four values calculated in the previous step agree
  with expectations and set passed accordingly if ALL passed.
\item The expected ATWD\_WAVEFORM\_HEIGHT depends on the ATWD channel
  (gain).  Assuming that ~2Volts is the entire ATWD range [0-1023] and
  that the gains are 8 counts/mV, 2 counts/mV and 0.3333 counts/mV for
  Channel 0, 1 and 2 respectively.  For instance if we have
  PULSER\_AMPLITUDE\_UVOLT at 5000 and we are looking at channel 1 of
  the ATWD, we expect a height of 5000 * 2 mV/count / 1000 = 10
  counts.
The measured height should be within 20% of the expected height.
\item The expected ATWD\_WAVEFORM\_BASELINE depends on whether
  pedestal subtraction was requested. If pedestal subtraction was
  requested, the value should be close to zero. If it was not
  requested, the value would depend on the gain and pedestal/analog
  reference settings. Therefore the condition will be limited to: If
  pedestal subtraction was requested, the ATWD\_WAVEFORM\_BASELINE
  should be consistent with zero to within 2 counts.
\item The expected ATWD\_WAVEFORM\_WIDTH depends on the true pulse
  width as determined from the time constants on the hardware
  components, the sampling speed and the bandwidth of the amplifiers.
  The result should be independent of all other setting and of the
  channel being used.  The measured ATWD\_WAVEFORM\_WIDTH should be
  within one sample of the expected width [this expected value needs
  to be filled in by the engineers or from initial oscilloscope pulser
  tests, but the 3.333 nsec/sample should be used as a fixed input
  parameter].
\item The expected ATWD\_WAVEFORM\_POSITION depends on the timing of
  pulser circuitry, the electrical length of the delay line, the
  sampling speed, the FPGA related delays and the requested delay
  between pulse and launch/trigger. The measured
  ATWD\_WAVEFORM\_POSITION should be within three samples of the
  expected value [this expected value needs to be filled in by the
  engineers or from initial measurements but it should assume 3.333
  nsec/sample and discrete additional delays from
  PULSE\_DELAY\_WRT\_LAUNCH of 25 nsecs/unit].
\end{itemize}

\subsection{Pulser ATWD0/1/2 previous pulse memory test Into FADC}

\subsection{Pulser ATWD inter-channel cross-talk test}

\subsection{Pulser SPE spectrum with discriminator?}

\subsection{PMT Coupling tests}

\subsection{Multiplexer to ATWD channel-3 tests}

\subsection{Clock sinusoidal wave (sampling speed)}

\subsection{Clock square 2x (sampling speed)}

\subsection{Voltage on On-board LED}

\subsection{Voltage from Flasher Board Interface}

\subsection{Local Coincidence Signal from Upper DOM}

\subsection{Local Coincidence Signal from Lower DOM}

\subsection{Signal from Communications Input Amplifier}

\subsection{Pulser Signal before Delay Line}

\subsection{Clock square 2x ATWD inter-channel cross-talk test }

\subsection{Communications test}

\subsection{Local Coincidence Hardware test}

\subsection{Clock Stability test}

\subsection{Communications ADC Noise test}

\subsection{Power Consumption test}

\subsection{Memory tests}

\subsection{Single Photoelectron Spectrum test}

\subsection{Cross-talk within board tests}

\subsection{Communications signals into PMT signal path cross-talk test}

\subsection{Transmitted signals}

Transmitted signals

\subsection{Received signals}

Received signals

\subsection{On-board LED electrical pulsing into PMT signal path cross-talk test}

\subsection{Flasher board electrical pulsing into PMT signal path cross-talk test}

\subsection{Local coincidence signals into PMT signal path cross-talk test}

\subsection{PMT signal into communications signal path cross-talk test}

\subsection{Pulser signal into communications signal path cross-talk test}

\subsection{On-board LED electrical pulsing into communications signal path cross-talk test}

\subsection{Flasher board electrical pulsing communications signal path cross-talk test}

\subsection{Local coincidence signals into communications signal path cross-talk test}

\subsection{Reading ADCs and pressure and temperature into PMT signal path cross-talk test}

\subsection{Reading ADCs and pressure and temperature into communications signal path cross-talk test }

\subsection{Two-DOM tests}

\subsection{Communications cross-talk }

\input{memory}



\section{Notes}

\begin{itemize}
\item Pulser DAC: A 10-bit DAC channel sets the pulser amplitude
  level. Its range is [0-5V], or 4.88 mV/LSB. To compute the
  $pulser\_dac$ use :

\[ pulser\_dac = 
Pulse\_in\_microvolts * 1024 * (R1+R2) / R2 / 5,000,000 \]

[where $R1$ and $R2$ are still not determined but should be around
$\frac{R1+R2}{R2} = 10$ to cover the 0.1-100 SPE range ]

\item Pedestal DAC:
  
  A 12-bit DAC channel sets the value of the pedestal. Its range is
  [0-5V].  The value is chosen to provide the ATWD with its ~3V target
  Vref. To compute the $pedestal\_dac$ use:

\[ pedestal\_dac = 
\frac{pedestal\_in\_microvolts \times 4096}{5,000,000} \]

(is this a problem for LONGS, or should I think in float
calculation?).

\item ADCs: The ADC (eight-channel unit) is a 12-Bit (0-4095) unit
  with an input range of 2.048 Volts.
 
\item PE amplitude: The nominal magnitude of a single photoelectron
  pulse out of the transformer is 5mV (mean). Before the
  discriminators there is a gain of ~9.6. Therefore the amplitude is
  ~48 mV (mean) for the SPE at the discriminator.
  
\item SPE/MPE DAC: A 10-bit DAC channel sets the SPE/MPE discriminator
  level. Its range is [0-5V], or 1.22 mV/LSB. To calculate the
  $spe\_dac$ use:


\[ spe\_dac =
(discriminator\_in\_microvolts \times 9.6 \times
\frac{2200+1000}{1000} + \frac{pedestal\_dac \times 5,000,000}{4096})
\cdots \]
\[ \times \frac{1024}{5,000,000} \]

\item To calculate the $mpe\_dac$

\[ mpe\_dac =
(discriminator\_in\_microvolts \times 9.6 + \frac{pedestal\_dac \times
  5,000,000}{4096}) \times {1024}{5,000,000} \]

\item Gains in the ATWD0,1,2 amplifiers: As of Jan 28 2003
\begin{itemize}
\item ATWD0 has x2x4.4
\item ATWD1 has 4.333333
\item ATWD2 has x2/3
\item ATWD3
\item ADC 1x4.88x4.88
\end{itemize}

\end{itemize}

\printindex

\end{document}
